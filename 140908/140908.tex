\documentclass[12pt,a4paper]{article}

\usepackage[utf8]{inputenc}
\usepackage[english,russian]{babel}
\usepackage{amsmath,amssymb,amsthm}
\usepackage{graphics}
\usepackage{ bbold }

\newcommand{\N}{\mathbb{N}}
\renewcommand{\le}{\leqslant}
\theoremstyle{plain}
\newtheorem{thm}{Теорема}
\newtheorem{lem}{Лемма}
\DeclareMathOperator{\mex}{mex}


\newcommand{\nim}[1]{*#1}



\begin{document}

\section{Игры на графах}
\subsection{Опредения}
Симметричной комбинаторной игрой с полной информацией на графе (далее просто \emph{игрой}) называется пара $<G(V, E), u>$ из ориентированного графа $G$ и его вершины $u$.

\emph{Ходом} в игре является замена вершины $u$ на такую вершину $v$, что в графе $G$ есть ребро из $u$ в $v$.

Проигравшим, считается игрок, который не может сделать ход.

Позиция называется \emph{выигрышной}, если игрок, который первым ходит в этой позиции может выиграть, вне зависимости от действий второго игрока.
Позиция называется \emph{проигрышной}, если игрок, который вторым ходит в этой позиции может выиграть, вне зависимости от действий первого игрока.
Иначе позиция называется \emph{ничейной}

\subsection{Выигрышно-проигрышное разбиение графа}

Введем семейства множеств вершин графа $L_{2k}$ и $W_{2k+1}$, $k \in \N\cup\{0\}$.

Определим эти множества по индукции.

$$L_0 = \{v \in V \mid \text{из }v \text{ нельзя сделать ход }\}$$
$$W_k = \{v \in V \mid \text{из }v \text{ есть ход в } L_{k-1}\}$$
$$L_k = \{v \in V \mid \text{из }v \text{ все ходы ведут в } W_{k-1}\}$$

Определим $W = \bigcup\limits_{k=0}^{\infty}{W_{2k+1}}$; $L = \bigcup\limits_{k=0}^{\infty}{L_{2k}}$

\begin{thm}
Множество $W$ "--- есть множество выигрышиных вершин графа.
Множество $L$ "--- множество проигрышных вершин.
Множество $D = V \setminus (W \cup L)$ "--- множество ничейных вершин графа.
\end{thm}

Будем считать, что проигрывающий игрок хочет максимально оттянуть поражение, а выигрывающий хочет выиграть как можно быстрее.

\begin{thm}
Множество $W_{2k+1}$ "--- есть множество вершин, в которых существует стратегия за первого игрока, приводящая к выигрышу за $2k+1$ ход.
Множество $L_{2k}$ "--- есть множество вершин, в которых существует стратегия за второго игрока, приводящая к выигрышу за $2k$ ходов.
\end{thm}

Обе теоермы тривиально доказываются по индукции.

Разбиение вершин графа на множества $W, L, D$ можно вычислить за время $O(E)$, используя следующий алгоритм
\begin{itemize}
\item Для каждой вершины будем хранить величину $z_v$, изначально равную исходящей степени вершины.
\item Будем хранить очередь из еще не обработанных вершин, про которые уже известно какому из множеств они принадлежат.
\item Изначально пометим как $L_0$ вершины, с исходящей степенью 0 и добавим их в очередь
\item Пока очередь не пуста, достаем из нее первый элемент (обозначим за $v$), и если 
\begin{itemize}
    \item $v \in L_{2k}$, добавим все еще не определенные вершины, из которых есть ребро в $v$ в множество $W_{2k+1}$ и положим их в очередь
    \item $v \in W_{2k+1}$, уменьшим на 1 величину $z_u$ для всех вершин из которых есть ребро в $v$. Те, у которых $z_u$ обнулилиось добавим в множеств $L_{2k+2}$ и в очередь.
\end{itemize}
\end{itemize}

Корректность алгоритма тривиально доказывается по индукции. Алгоритм работает за $O(E)$, так как каждая вершина будет обработана один раз,
а время обработки вершина равно ее входящей степени.

\subsection{Прямая сумма игр и эквивалентность}

\emph{Прямой суммой игр} $<G_1,~v_1>$ и $<G_2,~v_2>$ называется игра $<G_1~\times~G_2,~(v_1,~v_2)>$.

Можно это понимать, как игра, в которой можно сделать ход в одной из двух исходных игр.
Прямую сумму будем обозначать знаком $+$.

Несложно проверить, что по этой операции игры образуют моноид, нейтральным элементом в которой является игра
$\nim{0}$, устроенная как одна вершина, из которой нет переходов.

Обозначим 
$$
v(A) = \begin{cases}
-1, \text{если A "--- проигрышная} \\
~0, \text{если A "--- ничейная} \\
~1, \text{если A "--- выигрышная}\\
\end{cases}
$$

Назовем игры $A$ и $B$ \emph{эквивалентными}, если $v(A+C) = v(B+C)$ для любой игры $C$.
Будем обозначать это $A \cong B$.

\begin{thm}
Если $A_1 \cong A_2$, $B_1 \cong B_2$, то $A_1 + B_1 \cong A_2 + B_2$.
\end{thm}

Эта теорема, говорит, что операция прямой суммы согласована с определнной эквивалентностью.

\subsection{Теория Гранди для ацикличных игр}

\begin{thm}
Если $A$ "--- игра на ацикличном графе, то $A + A$ "--- проигрышна.
\end{thm}

\begin{thm}
Если $L$ "--- проигрышная игра, то $v(L + C) = v(C)$.
\begin{proof}
Разбор случаев результата игры C, и явным построением стратегии.
\end{proof}
\end{thm}

\begin{thm}
Если $L$ "--- проигрышная игра, то $L \cong \nim{0}$.
\end{thm}

В частности $A + A = \nim{0}$, то есть классы эквивалентности игр на ацикличных графах образуют абелеву группу.

Игрой \emph{ним} размера $k$ (обозначатеся $\nim{k}$) называется игра, из которой есть переходы
во все нимы с меньшими номерами.

\begin{thm}
Если $i \neq j$, то $\nim{i} \ncong \nim{j}$.
\begin{proof}
Разный результата в сумме с $\nim{\min(i, j)}$.
\end{proof}
\end{thm}

\begin{thm}
$A \cong \nim{i} \Leftrightarrow v(A+\nim{i}) = -1$.
\end{thm}

Определим функцию $\mex$ от множества неотрицательных целых чисел,
как наименьшее неотрицательное целое число, которое не лежит в множестве.

\begin{thm}
Если $A$ "--- игра на ацикличном графе, то $A \cong \nim{g(A)}$, для некоторого числа $g(A)$,
причем, если из $A$ есть переходы в игры, эквивалентные $\nim{g(A_1)}, \nim{g(A_2)}, \dots \nim{g(A_k)}$,
то $g(A) = \mex\{g(A_1), g(A_2), \dots g(A_k)\}$
\begin{proof}
Доказательство индукцией по длине пути до самой далекой терминальной вершины.
\end{proof}
\end{thm}

Число $g(A)$ называется \emph{функцией Гранди} игры $A$. Теорема дает конструктивный 
способ вычисления функции гранди за $O(E)$.

За $\oplus$ обозначим побитовый ксор двух чисел.

\begin{thm}
$g(A+B) = g(A) \oplus g(B)$
\begin{proof}
Явное вычисление $\mex$ для нимов.
\end{proof}
\end{thm}

\subsection{Теория Смита}

\begin{thm}
Пусть $A$ "--- игра на графе. Пусть из $A$ есть переходы в игры $A_1, \cdots A_k$, причем 
$g$ "--- минимальное неотрицательное целое число, такое что ни одна из игр не эквивалентна $\nim{g}$.
Пусть для каждой игры $A_i$ выполнено либо
\begin{itemize}
\item $A_i$ эквивалентно ниму
\item из $A_i$ есть переход в игру, эквивалентную $\nim{g}$
\end{itemize}
Тогда $A \cong \nim{g}$.
\end{thm}

Рассмотрим следующий алгоритм. 
Пока есть вершина, для которой можно найти эквивалентный ей ним, используя теорему,
отметим ее как эквивалентную ниму.
Если вершина не отмечена алгоритмом, назовем ее \emph{неопределнной}

\begin{thm}
Неопределенная вершина не эквивалентна никакому ниму.
\begin{proof}
Рассмотрим игру, которая в сумме с нимом проиграна за минимальное число ходов.
\end{proof}
\end{thm}

\begin{thm}
Пусть существует игра $B$, такая что $A + B \cong \nim{0}$. Тогда 
$A \cong \nim{i}$ для некоторого i.
\begin{proof}
Индукция по числу ходов до проигрыша в $A+B$
\end{proof}
\end{thm}

\begin{thm}
Если вершина неопределнная, то соответсвующая ей игра выигрышная или ничейная в сумме с любой другой.
\end{thm}

Для множества $K = \{g_1, g_2, \dots, g_k\} $ обозначим за $\infty_K$ игру из которой есть переходы
в игры $\nim{g_1}, \nim{g_2}, \dots \nim{g_k}, \infty_K$
Игра $\infty_K$ "--- выигрышная, если $0 \in K$, и ничейная иначе.

\begin{thm}
Каждая игра эквивалентна либо некоторому ниму, либо некоторой игре $\infty_K$, причем
\item $\nim{i} + \nim{j} \cong \nim{i \oplus j}$
\item $\nim{\infty_K} + \nim{i} \cong \infty_{K \oplus i}$
\item $\nim{\infty_K} + \nim{\infty_L} \cong \infty_\varnothing$
\end{thm}

Данная теорема полностью описывает вид моноида на играх. Игру, которой эквивалентна каждая вершина графа можно определить за время $O(E\sqrt{E})$,
определяя игры эквивалентные каждому ниму по очереди, алгоритмом аналогичным ретроанализу. Функция гранди вершины не привосходит $\sqrt{2E}$, что оценивает время работы.






\end{document}                                
