\documentclass[10pt,a4paper]{article}
%\hoffset=-35mm
%\voffset=-35mm
%\textheight=270mm
%\textwidth=175mm



\usepackage[utf8]{inputenc}
\usepackage[russian]{babel}
\usepackage{amsmath,amssymb,amsthm}
\usepackage{graphicx,amscd,color}
\usepackage{graphics}

\newtheorem{thm}{Теорема}


\begin{document}

\raggedbottom

\section{Постановка задачи.}

Дан граф $G = (V, E)$. Необходимо перевести через реку 
$|V|$ предметов, не оставляя предметы соединенные ребром,
без присмотра, имея лодку размера $b$.
Формально, планом перевозки для лодки размера $k$ называется последовательность
$(L_{i}, R_{i}, B_{i}), i = 1..s$, такая что
\begin{itemize}
\item $L_i \sqcup R_i \sqcup B_i = V$
\item $L_i, R_i$ "--- независимые множества в $G$
\item $L_1 \cup B_1 = V$
\item $R_{s} \cup B_{s} = V$
\item $L_{k} = L_{k-1}$ для четных $k \ge 2$
\item $R_{k} = R_{k-1}$ для нечетных $k \ge 3$
\item $|B_i| \le b$.
\end{itemize}

Обозначим минимальное $b$ для которого существует план перевозки $AN(G)$.

\section{Структурная теорема}

Пусть $VC(G)$ "--- размер наименьшего вершинного покрытия в $G$.

Легко проверить, что $VC(G) \le AN(G) \le VC(G)+1$.

Мы будем использовать следующие факты, доказанные в статье (ссылка).

\begin{thm}
План для размера $b$ существует тогда и только тогда, 
когда существует разбиение $V = X_1 \sqcup X_2 \sqcup X_3 \sqcup Y$,
а также два непустых подмножества $Y_1, Y_2 \subset Y$, таких что
\begin{itemize}
\item $X_1 \cup Y_1$, $X_2 \cup Y_2$, $X_1 \cup X_2 \cup X_3$ "--- независимые
\item $|Y| \le b$
\item $|Y_1| + |Y_2| \ge |X_3|$
\end{itemize}
\end{thm}

\begin{thm}
Если в графе есть хотя бы два минимальных вершинных покрытия,
то $AN(G) = VC(G)$.
\end{thm}

\newpage

\section{Алгоритм для решения из статьи}

\begin{itemize}
\item Найдем в графе $G$ минимальное вершинное покрытие $Y$.
\item Если $b < |Y|$, то вернуть <<NO>>.
\item Если $b \ge |Y| + 1$ то вернуть <<YES>>.
\item Если в графе $G$ есть другое вершинное покрытие (не сложно проверить за b поисков вершинного покрытия),
то вернуть <<YES>>.
\item Переберем $Y_1, Y_2 \subset Y$.
\item В качестве $X_1$ возьмем вершины $V\setminus Y$ не связные с $X_1$, в качестве $X_2$ "--- вершины $V \setminus (Y \cup X_1)$ 
не связные с $Y_2$. В качестве $X_3$ все вершины $V \setminus (Y \cup X_1 \cup X_2)$.
\item Проверим для полученных множеств утверждение теоремы. Если оно выполнено, вернуть <<YES>>.
\item Если для всех $Y_1, Y_2 \subset Y$ утверждение не выполнено вернуть <<NO>>
\end{itemize}

Итоговая сложность алгоритма $O^*(b\cdot T_{VC}(b) + 4^b)$, где $T_{VC}$ "--- время на решение задачи о вершинном покрытии размера $k$.

\section{Улучшенный алгоритм}

Будем пытаться найти оптимальное $Y_2$ за время быстрее, чем $2^{k}$.

После перебора $Y_1$ удалим из графа все вершины $X_1$.
Обозначим новый граф $G' = (V', E')$, .

Посмотрим на множество $Z = X_3 \cup (Y \setminus Y_2)$.
Оно является вершинным покрытием в графе $G'$ как дополнение 
независимого $Y_2 \cup X_2$. Причем между множествами $Z$ и тройками 
($Y_2$, $X_2$, $X_3$) есть естественная биекция.
Поэтому, можно вместо выбора $Y_2$ выбирать $Z$, при этом
автоматически будут выполнены все условия, кроме условия на размеры $Y_1, Y_2, X_3$,
и условия на непустоту $Y_2$.
Перепишем условие на размер в терминах $Z$.

$$|Y_1| + |Y_2| - |X_3| = |Y_1| + |Y| - |Y \setminus Y_2| - |X_3| =
b + |Y_1| - |Z| \ge 0$$.

То есть задача выбора $Z$ "--- это поиск вершинного покрытия в графе $G'$ размера не более $b + |Y_1|$,
отличного от $Y'$. Для решения этой задачи
переберем вершину $v \in Y'$ которая не войдет в покрытие.
Тогда мы обязаны взять всех ее соседей, после чего надо
найти вершинное покрытие в графе $G \setminus (\{v\} \cup N_G(v) \cup X_1)$, размера меньше чем $b + |Y_1| - N_G(v)$.
Заметим, что если $b + |Y_1| - N_G(v) \ge b - 1$, то такое покрытие точно найдется "---
$Z = Y \setminus \{v\}$, поэтому можно считать, что параметр задачи 
не превосходит $b$. 
Таким образом, задача решается за $O^*(b\cdot T_{VC}(b))$.

Суммарная сложность алгоритма $O^*(2^b\cdot b\cdot T_{VC}(b))$.

Известно, что вершинное покрытие можно решать за время $T_{VC}(b) = O^{*}(1.2738^b)$,
в таком случае время работы алгоритма $O^*(2.5476^b\cdot b)$

\section{Алгоритм за экспоненту от размера графа}

\subsection{За $4^n$}

Найдем вершинное покрытие $Y$ в графе. Если оно не единственно,
то все понятно.

Иначе, рассмотрим его дополнение $X$. Переберем
все пары $X_1, X_2 \subset X, X_1 \cap X_2 = \varnothing$.

За $Y_1$ и $Y_2$ возьмем максимальное подмножества $Y$, такие, что
$X_1 \sqcap Y_1, X_2 \sqcap Y_2$ независимы.
Проверим для этого набора множеств условие теоеремы.

\subsection{За $4^{n-b} + 2^b$}

За $2^b$ можно для всех подмножеств $Y$ найти наибольшее независимое подмножество,
с помощью динамического программирования.

Тогда вторую часть можно выполнять за $O^*(1)$ "--- надо взять самое большое независимое подмножество,
среди эелементов Y, не соединенных ни с кем из $X_1$ или $X_2$.

\subsection{за $2^{n-b} + 2^b$}

Переберем все $X' \subset X$.
Для каждого $X'$, для которого есть непустое $Y_{X'} \subset Y$,
такое что $X' \sqcup Y_{X'}$ независимо, выбрем максимальный $Y_{X'}$ по размеру
и запишем моном $x^{bits(X') + 2^{n-b} |Y| + 2^{n - b + logb + 1} |X'|}$,
где $bits(X')$ "--- битовая маска длины $n-k$, соответствующая множеству.

Сложим все эти мономы. Мы получили многочлен степени $O^*(2^{n-b})$.
Возведем его в квадрат за $O^*(2^{n-b})$ с помощью
быстрого преобразования Фурье.

Рассмотим все мономы имеющие вид $x^{bits(X') + 2^{n-b}t + 2^{n - b + logb + 1} |X'|}$.
Они могли быть получены только как произведение двух мономов соответствующих
непересекающимся $X'$ (т.к размер сошелся).
Среди всех таких необходимо выбрать моном с максимальным $t - (n - |X'|)$.
Если эта величина больше 0, то $AC(G) = VC(G)$, иначе $AC(G) = VC(G)+1$.
В качестве множеств $X_1$ и $X_2$ следует взять те, из которых был получен данный моном.

\subsection{за $2^{n-b} \cdot VC^b$}

Все тоже самое, только не предподсчитываем размеры, а решаем
задачу для каждого множества отдельно.

\subsection{Комбинируем!}

Если $b \le 0.425 n$, то воспользуемся алгоритмом за 
$2.5476^b \le 2.5472^{0.425n} \le 1.488^n$.


При $0.425 n \le b \le 0.605n$ воспользуемся алгоритмом за $2^{n-b} + 2^b$.
$max(2^{n-b}, 2^b) \le 2 ^{0.605n} \le 1.52^n$.

При $b \ge 0.605n$ воспользуемся алгоритмом за $2^{n-b}* VC^b$
Оно убывает $2^{n - b} \cdot VC^b \le 2^{(1 - 0.605)n} \cdot 1.2738^{0.605n} = 1.521^n$.



\end{document}                                
