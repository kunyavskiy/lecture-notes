\documentclass[10pt,a4paper]{article}
%\hoffset=-35mm
%\voffset=-35mm
%\textheight=270mm
%\textwidth=175mm



\usepackage[utf8]{inputenc}
\usepackage[english,russian]{babel}
\usepackage{amsmath,amssymb,amsthm}
\usepackage{graphicx,amscd,color}
\usepackage{graphics}

\newcommand{\Cl}{\operatorname{cl}}
\newcommand{\N}{\mathbb{N}}
\newcommand{\R}{\mathbb{R}}


\begin{document}

\raggedbottom


Заметим, что от эвристики B ничего не зависит, так как все равно необходимо проверять обе ветви рекурсии.
Суммарное количество рекурсивных вызовов не зависит от порядка проверки.

Рассмотрим дерево рекурсивных вызовов.
Докажем, что в нем нет листьев глубины менее $n-2$. Глубиной корня считается 0. В таком случае в 
нем не менее $2^{n-2}$ вершин. $2^{n-2} > 2^{\frac{n}{2}}$ для достаточно больших $n$,
что дает нужную оценку.

Вершина является листом, если после применения правил упрощения,
обнаружится противоречие, т.е. дизъюнкт в котором все значения в подстановке
уже установлены как <<0>>.

Докажем по индукции следующее утверждение:
пусть $k <= n-2$. Тогда при рекурсивном вызове глубины $k$ после применения правил упрощения
\begin{enumerate}
\item Для любого $i$ либо есть переменная $p_{i,j}$ со значением <<1>>, или есть не более чем $k$ переменных $p_{i,j}$ со значением <<0>> в подстановке $\rho$.
\item Для любого $j$ либо есть переменная $p_{i,j}$ со значением <<1>>, или есть не более чем $k$ переменных $p_{i,j}$ со значением <<0>> в подстановке $\rho$.
\item Не более чем $k$ переменных $p_{i,j}$ имеют значение <<1>> в подстановке $\rho$
\end{enumerate}

{\bf База.} При первом запуске нет ни одной переменной имеющей значение в $\rho$.

{\bf Переход.} Рассмотрим два случая. 
\begin{enumerate}
\item Если для переменной значение которой установлено на предыдущем уровне выбрано значение <<1>>.
	Это позволит для каждого $i$ установить не более чем еще одной переменной $p_{i,j}$ значение <<0>>, используя
	правило удаление единичного дизъюнкта --- c тем $j$ для которой одноа из $p_{s,j}$ была установлена в <<1>>.
	При этом таких переменных станет не более чем $n-2$, что не позволит применить правило чистых литералов,
	чтобы установить значение $p_{i,t}$ <<1>> для какого-нибудь $t$.
	Аналогично Это позволит для каждого $j$ установить не более чем еще одной переменной $p_{i,j}$ значение <<0>>.
	При этом по прежнему, не получится установить какой-либо переменной значение <<1>>.
	Таким образом в этом случае переход доказан.
\item Если для переменной значение которой установлено на предыдущем уровне выбрано значение <<0>>.
	В этом случае, не получится применить ни одно из правил. При этом утверждение индукции сохраняется.
\end{enumerate}

Таким образом, противоречия в дизъюнктах $p_{i,1} \vee p_{i,2} \vee \dots \vee p_{i,n}$ получить на вызовах
глубины не более чем $n-2$ не получится, так как не более $n-2$ переменным будут выбрано значение <<0>>.
Получить противоречие с $\overline{p_{i,j}} \vee \overline{p_{k,j}}$ также не получится.
Для того, чтобы получить это противоречие, надо обеим переменным установить значение <<1>>.
Это может произойти только в результате применения правила удаления единичного дизъюнкта,
к одной из переменных. Но для этого надо чтобы было $n-1$ переменная с значением <<0>> либо
среди $p_{i,*}$, либо среди $p_{j,*}$, что невозможно при глубине рекурсивных вызовов не более
$n-2$. 

\end{document}                                
